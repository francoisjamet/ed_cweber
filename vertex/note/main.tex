\documentclass{book}
\usepackage[utf8]{inputenc}
\usepackage[T1]{fontenc}
\usepackage{fixltx2e}
\usepackage{graphicx}
\usepackage{longtable}
\usepackage{siunitx}
\usepackage{float}
\usepackage[french,greek,english]{babel}
\usepackage{wrapfig}
\usepackage{rotating}
\usepackage[normalem]{ulem}
\usepackage{amsmath,amssymb}
\usepackage{textcomp}
\usepackage{marvosym}
\usepackage{wasysym}
\usepackage{amssymb}
\usepackage{hyperref}
\tolerance=1000
\usepackage[margin=3cm]{geometry}
\usepackage[backend=bibtex,sorting=none]{biblatex}
\addbibresource{bibli.bib}  %% point at your bib file
\usepackage{caption}
\usepackage{subfigure}

\def\a{\alpha}
\def\hx{\hat{x}}
\def\b{\beta}
\def\D{\Delta}
\def\d{\delta}
\def\e{\epsilon}
\def\g{\gamma}
\def\G{\Gamma}
\def\h{\eta}
\def\k{\kappa}
\def\l{\lambda}
\def\S{\Sigma}
\def\v{\nu}
\def\s{\sigma}
\def\sch{Schr\"{o}dinger }
\def\t{\tau}
\def\K{\si{\kelvin}}
\def\w{\omega}
\def\dr{\partial}
\def\dd{\dagger}
\def\ua{\uparrow}
\def\da{\downarrow}
\def\dtau{\varDelta \tau}
\def\Vec#1{\mathbf #1}
\def\u#1{\underline #1}
\def\wh#1{\widehat #1}
\def\mean#1{\left< #1 \right>}
\DeclareMathOperator{\Tr}{Tr}
\usepackage{listings}
\lstset{
  language=bash,
  basicstyle=\ttfamily
}


\begin{document}

\section{Installation}


You need to install first the solver and then a small extension for the susceptibility calculation.

I would recommend to use my version of Cedric's solver.

In the main directory, change the file Makefile.in with your configuration. In principle, the only thing which needs to be changed is the lapack/blas library. In my case, I use openblas which can be easily installed on any GNU/Linux system.

Then run ./run\_install.sh

It prints many warnings... but should works and create the folder bin which contains the main executables.


\section{ED solver}

\subsection{PARAMS}

The PARAMS file contains all the information about the impurity except the hybridization $\Delta$.

The format is :
\begin{lstlisting}
n_imp ! number of site in the impurity ( For the rest of this example, I assume that n_imp=2.
E_11_up E_12_up
E_21_up E_22_up
E_11_dn E_12_dn
E_21_dn E_22_dn
U_11 U_12
U_21 U_22
nomg! number of matsubara frequency
nomg ! number of matsubara frequency
nomg_real ! number of point on the real axis
F ! keep it unless you know what you are doing
1 ! not used
F ! T to compute the Green function on the real axis
0 ! wmin for real axis
0 ! wmax for real axis
0 ! Chemical potential (equivalent to shift E to E-mu
100.0 ! Inverse temperature
0 ! Do not touch any numbers after this one unless you know what you are doing.
 5.0000000000000001E-004
   1.4000000000000000E-002
   1.2999999999999999E-002
   3.5999999999999997E-002
   10
0.0
 F
F
F
F
0.0000
\end{lstlisting}



\subsection{$\D(\w)$}

the hybridization is contained in the files delta1.inp (spin up) and delta2.inp.

the format is :

\begin{align}
  \w_1 \, \Re(\D_{11}) \,  \Im(\D_{11})\,  \Re(\D_{12}) \, \Im(\D_{12})\,  \Re(\D_{21}) \, \Im(\D_{21}) \, \Re(\D_{22}) \, \Im(\D_{22}) \nonumber\\
  \w_2 \, \Re(\D_{11}) \,  \Im(\D_{11})\,  \Re(\D_{12}) \, \Im(\D_{12})\,  \Re(\D_{21}) \, \Im(\D_{21}) \, \Re(\D_{22}) \, \Im(\D_{22}) \nonumber\\
  .... \nonumber
\end{align}




\section{Susceptibility}


The difficulty of susceptibility calculation in ED is that transition between two nearby igh energy states  will contribute. That is why Lanczos is not a good option and only Full\_ed works.

So the susceptibility calculation is limited to number site+bath <= 8.

To compute the suscpetibility :

\begin{enumerate}
\item modify ED/ED.im
  \begin{itemize}
  \item FLAG\_DUMP\_INFO\_FOR\_GAMMA\_VERTEX=.true.
  \item Neigen=1000 !(this is for $n_{bath}+n_{imp} = 8$, it could be decrease if this number is smaller.
  \item dEmax0=100000 !  This number controles the energy of higher energy state included. for a normal calculation, it has to be small in order to speed up the calculation, but in our case, we want as much as possible states.
  \item  Nitergreenmax=1 ! we do not want to compute the green fct, it would be too expensive with  dEmax0 large
  \end{itemize}

\item Run dmft\_solver as usual

\item use omega\_path $N$ where $N$ is the number of matsubara fermionic frequencies. It creates the file omega\_list\_path.

\item create the file cutoff with contains 4 numbers :

\begin{lstlisting}
  1            # first bosonic frequency to compute
  3            # last bosonic frequency to compute
  1d-5
  1d-9
\end{lstlisting}
The last two numbers are cutoffs used in the code.
In practice, dmft\_chiloc will  compute this formula :

\begin{equation}
  \chi = \sum_{ijkl} e^{-\b E_i} \Phi(E_i,E_j,E_k,E_l,\v_1,\v_2,\O) c_{ij}c_{jk}c_{kl}c_{li}
\end{equation}

where $E_i$ is the energy level of state i,  $c_{ij}$ a matrix transition between state $ij$, and  $ \Phi(E_i,E_j,E_k,E_l,\v_1,\v_2,\O)$ a function which depends of all the fermionic and bosonic frequencies.

The number of element of this sum increases as $N_{eigen}^3$ ! It is crucial to reduce it as much as we can and that is the role of the cutoffs

The element is computed if $e^{-\b E_i} >cutoff_1$ and $c_{ij}c_{jk}c_{kl}c_{li} >cutoff_2$.


Last but not least, dmft\_chiloc works with mpi and openmp. Use as much threads as cpu available  and mpi to share between nodes.


The output of the dmft\_chiloc is a series of files which contains the susceptibility. To convert them in a nice hdf5 file format, run the program readvertex

readvertex takes 2 arguments : nomb beta

The first one is the number of bosonic frequency and beta the inverse temperature.

It creates the file chiS.h5 which contains 'chis' and 'chic' (charge and spin susceptibilities).

there are both arrays $A[iv1,iv2,i1,i2,i3,i4,iom]$,

iv1 and iv2 fermionic frequency index

i1,i2,i3,i4 orbital index

iom bosonic freqnency index
\end{enumerate}


A simple python program to plot the susceptibility could be  :

\begin{lstlisting}
import h5py
import matplotlib.pyplot as plt
import numpy as np

f = h5py.File('chiS.h5','r')
cs = np.array(f['chis'])
cc = np.array(f['chic'])
i1,i2,i3,i4 = 0,0,0,0
iom = 0
plt.plot(np.diag(cs[:,:,i1,i2,i3,i4,iom].real))
plt.show()
\end{lstlisting}

\end{document}
